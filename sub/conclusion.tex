%-----------------------------------------------------
% Conclusion
%-----------------------------------------------------
\newpage

\section{Conclusion}
본 연구의 결과는 다음과 같다:
\begin{enumerate} 
	\item
	
	Exciton recombination rate와 biexciton recombination rate 사이의 비율의 평균은 3.43와 3.30이었다. 이론적으로 알려진 2.29이라는 비율\cite{chen2018room} 과는 다소 상이하였으나 두 속도 사이의 비율이 결정의 위치가 바뀌어도 일정한 것으로 보아 만들어진 결정이 동일한 특성을 지닌 물질이라는 것을 확인 할 수 있었다. 두 값이 약간의 차이를 갖는 이유는 2.29라는 값은 $\rm{CsPbBr_3}$ 박막의 값이지만 3.30나 3.43라는 값은 $\rm{CsPbBr_3}$ 단결정의 값이기 때문이라고 예상된다.  
	\item 결정의 중심부로 갈수록 커지는 exciton / biexciton recombination rate으로부터 결정의 중심부의 결정의 순도가 더 높다는 것을 알 수 있다. recombination rate가 커진다는 것은 원자가띠에 인접한 defect의 에너지띠보다 전도띠에서 더 많은 recombinantion이 일어났다는 의미이다. 그렇기 때문에 recombination rate가 증가하면, defect가 적다는 의미이고, 순도가 높다는 의미이기도 하다.
\end{enumerate}
결국 PDMS-stamping 방법으로 만든 $\rm{CsPbBr_3}$ 단결정이 기존의 $\rm{CsPbBr_3}$와 동일하다는 것을 보일 수 있었다. 또, 결과 2번 항목에서 결정의 순도의 비교는 오직 정성적으로 이루어졌다. 만약 recombination rate와 결정의 순도 간의 정량적인 관계를 구할 수 있다면 결정의 순도를 유추할 수 있는 획기적인 방법이 될 것이다.
마지막으로 박막 형태의 $\rm{CsPbBr_3}$와 단결정 $\rm{CsPbBr_3}$에서의 Exciton recombination rate과 biexciton recombination rate 사이의 비율이 약간의 차이를 갖는 이유에 대해서는 추가적인 실험과 연구ur가 필요할 것으로 보인다. 
