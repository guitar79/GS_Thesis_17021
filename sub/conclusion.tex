%-----------------------------------------------------
% Conclusion
%-----------------------------------------------------
\newpage

\section{Conclusion}
본 연구의 결과는 다음과 같다:
\begin{enumerate} 
	\item PDMS stamping 방식으로 만들어진 결정은 광학 현미경을 통해 육안으로 확인하였고, X선 회절법을 이용하여 이 결정이 기존의 결정과 같은 X선 회절 무늬를 가지고 있음을 확인 할 수 있었다.
	\item	
	Exciton recombination rate와 biexciton recombination rate 사이의 비율의 평균값은 ND 0 filter 일 때는 3.43, ND 1 filter일 때는 3.30이었다. 이론적으로 알려진 2.29라는 비율\cite{chen2018room} 과는 다소 상이하였으나 두 재결합률 사이의 비율이 결정의 위치가 바뀌어도 일정한 것으로 보아 만들어진 결정이 동일한 특성을 보인 물질이라는 것을 확인 할 수 있었다. 두 값이 약간의 차이를 갖는 이유는 2.29라는 값은 $\rm{CsPbBr_3}$ 박막의 값이지만 3.30 나 3.43이라는 값은 $\rm{CsPbBr_3}$ 단결정의 값이기 때문이라고 예상된다.  
	\item 결정의 중심부로 갈수록 커지는 exciton / biexciton recombination rate로부터 결정의 중심부 결정의 순도가 높아진다는 경향성을 찾을 수 있었다. recombination rate가 커진다는 것은 원자가띠에 인접한 defect의 에너지띠보다 전도띠에서 더 많은 recombination이 일어났다는 의미이다. 그렇기 때문에 recombination rate가 증가하면, defect가 적다는 의미이고, 순도가 높다는 의미이기도 하다.
	\item TRPL 데이터의 분석을 통해 Figure 8의 pt 10은 중심부로 갈수록 오히려 exciton / biexciton recombination rate가 감소하는 양상을 보여주었는데, 이는 중심부로 갈수록 순도가 감소한다는 것을 의미하는 것이 아니라 이 부분의 결정이 불완전하게 생성되었음을 의미한다. 즉, PDMS-stamping의 시간을 연장하거나 온도 조건을 조절함으로써 pt 10의 결정을 완성한다면 예측과 같이 순도가 증가할 것으로 예상된다. 
\end{enumerate}
PDMS-stamping 방법으로 만든 $\rm{CsPbBr_3}$ 단결정이 기존의 $\rm{CsPbBr_3}$와 동일하다는 것은 X선 회절 분석으로 보일 수 있었다. 하지만, TRPL 분석을 통해 보존되는 비율로 만들어진 결정이 $\rm{CsPbBr_3}$임을 보이려는 시도는 성공적이지 못했다. 그렇기 때문에 단결정이 만들어졌다는 결론은 X선 회절 분석으로 내린 후, TRPL 분석 결과로는 exciton recombination rate 와 biexciton recombination rate사이의 비율이 PDMS-stamping 방법으로 만든 단결정이 더 크게 관찰되었기 때문에 이 단결정이 기존의 단결정보다 exciton의 recombination이 활발하게 일어난다는 것을 유추하는 것이 올바른 해석이었다.  
또, 결과 3번 항목에서 결정의 순도의 비교는 오직 정성적으로 이루어졌다. 그렇기 때문에 만약 recombination rate와 결정의 순도 간의 정량적인 관계를 구할 수 있다면 결정의 순도를 유추할 수 있는 획기적인 방법이 될 것이다.
마지막으로 박막 형태의 $\rm{CsPbBr_3}$와 단결정 $\rm{CsPbBr_3}$에서의 exciton recombination rate과 biexciton recombination rate 사이의 비율이 약간의 차이를 갖는 이유에 대해서는 추가적인 실험과 연구가 필요할 것으로 보이고, exciton이 우세한 $\rm{CsPbBr_3}$가 필요한 분야에 대한 탐색이 이루어져야 할 것이다. 


