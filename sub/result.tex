\section{Results}


\subsection{광학 현미경 관찰}
단결정이 만들어졌는지 확인하기 위하여 현미경으로 관측한 결과를 figure \ref{fig:FIR103}의 사진들로 확인할 수 있다.
\begin{figure}[h!]
	\begin{center}
		\begin{tabular}{ccc}
			\includegraphics[height=3.5cm]{optic} &   \includegraphics[height=3.5cm]{crystal}&
			\includegraphics[height=3.5cm]{crystal_made}
		\end{tabular}
	\begin{tikzpicture} [remember picture,overlay]
	\node[text=white] at (-13.7, 1.5) {(a)};
	\node at (-8.7, 1.5) {(b)};
	\node[text=white] at (-4.3, 1.5) {(c)};
	\end{tikzpicture}
		\caption{(a) The picture shows the silicon wafer with no crystals. After the crystal successfully grew, it could be seen by the eyes :(b), and by optic microscopy :(c).}	
		\label{fig:FIR103}
	\end{center}
\end{figure}
\subsection{X-Ray 회절 분석}
$\rm{CsPbBr_3}$의 XRD peak는 2θ = 14.919$^{\circ}$ , 30.099$^{\circ}$ , 47.957$^{\circ}$에서 발견되는 기존의 데이터와 일치하기 때문에 XRD 분석을 통해 $\rm{CsPbBr_3}$ 결정이 만들어졌다는 것을 알 수 있다\cite{rakita2016low}.\\
\begin{figure}[H]
	\begin{center}
		\begin{tabular}{c}
			\includegraphics[width=0.45\textwidth]{XRD}
		\end{tabular}
		\caption{The angle of incidence was varied from 10° to 70°. The graph was drawn with Origin 8.0. Its (101) face is parallel to the scanning plane in a $\theta$/2$\theta$ scan. }	
		\label{fig:FIR104}
	\end{center}
\end{figure}


\subsection{TRPL 분석}
TRPL 그래프는 figure \ref{fig:FIR105} 와 같다. t = 28 ns부근에서 최댓값을 확인 할 수 있었고, 그 시간 이후의 데이터를 시간에 따른 지수 함수(exponential function)들의 합으로 fitting 할 수 있는데, 그 식은 $\sum_{i}^{} {e}^{-t/{\tau}_{i}}$ 로 표현된다. $\rm{CsPbBr_3}$는 exciton과 biexciton의 recombination으로 나뉘기 때문에 두 개의 exponential function의 합으로 표현하였다. 그래프의 피팅을 위해 파이썬을 활용하였다. TRPL 데이터를 순서쌍으로 바꿔서 그래프를 그린 후, mathplot library에 내재된 함수인 curve fit을 이용하여 최소제곱법으로 가장 비슷한 함수를 찾아낸다.
\begin{figure}[h]
	\begin{center}
		\begin{tabular}{c}
			\includegraphics[width=14cm]{TRPL_graph}
		\end{tabular}
		\caption{TRPL graph was fitted with sum of exponential functions. Two pictures at the right top shows the point where the TRPL datas were collected. Left picture corresponds to ND 0 filter,and the right picture corresponds to ND 1 filter. }	
		\label{fig:FIR105}
	\end{center}
\end{figure}
 만들어진 결정이 $\rm{CsPbBr_3}$ 임을 확인하기 위해서 exciton과 biexciton의 recombination rate에 대한 비율을 계산하였다. ND 0 필터를 사용했을 때 두 recombination rate 사이의 비율에 대한 평균값은 3.43 이었고 ND 1필터를 사용 했을 때는 3.30이었다.
\begin{figure}[t]
	\begin{center}
		\begin{tabular}{cc}
			\begin{tikzpicture}
			\begin{axis} [
			width=0.50\textwidth,%
			height = 6cm,%
			ybar,%
			title={ND 0 filter},%
			xtick = data,%
			symbolic x coords={pt1, pt2, pt3, pt4, pt5, pt6, pt7, pt8},%
			ylabel= {ratio},%
			ymin=0,ystep=0.5,ymax=10.0,%
			scaled y ticks = false,%
			ymajorgrids = true,
			legend style={at={(0.02,10)}},legend pos=north west]%
			\addplot table [x=pt, y=data] {./pt_data/ratio_nd0.csv}; %\addlegendentry {2003 LAC}%
			\end{axis}
			\node at (-0.2, 5.0) {(a)};
			\end{tikzpicture}
			&
			\begin{tikzpicture}
			\begin{axis} [
			width=0.50\textwidth,%
			height = 6cm,%
			ybar,%
			title={ND 1 filter},%
			xtick = data,%
			symbolic x coords={pt1, pt2, pt3, pt4, pt5, pt6, pt7, pt8, pt9, pt10, pt11},%
			ylabel= {ratio},%
			ymin=0,ystep=0.5,ymax=10.0,%
			scaled y ticks = false,%
			ymajorgrids = true,
			legend style={at={(0.02,10)}},legend pos=north west]%
			\addplot table [x=pt, y=data] {./pt_data/ratio_nd1_2.csv}; %\addlegendentry {2003 LAC}%
			\end{axis}
			\node at (-0.2, 5.0) {(b)};
			\end{tikzpicture}	
		\end{tabular}		
		\caption{The histogram shows the ratio between exciton recombination rate and biexciton rate, differed by the point of laser. (a) is when ND 0 filter is used, and (b) is when ND 1 filter is used. The standard deviation of ratio is 0.432 and 0.483, respectively. }	
		\label{fig:FIR106}
	\end{center}
\end{figure}
\begin{figure}[h]
	\begin{center}
		\begin{tabular}{ccc}
			\begin{tikzpicture}
			\begin{axis} [
			width=0.5\textwidth,%
			height = 5cm,%
			ybar,%
			bar width=10pt,
			title={ND 0 filter},%
			xtick = data,%
			symbolic x coords={pt6, pt4},%
			ylabel= {nsec},%
			ymin=0,ystep=0.2,ymax=2.5,%
			scaled y ticks = false,%
			ymajorgrids = true,
			legend style={at={(0.02,10)}},legend pos=north west]%
			\addplot table [x=pt, y=tau1] {./ND_data/nd0_1.csv}; \addlegendentry {tau 1},%
			\addplot table [x=pt, y=tau2]
			{./ND_data/nd0_1.csv}; \addlegendentry {tau 2}%
			\usetikzlibrary{patterns},
			\end{axis}
			\node at (-0.2, 4.0) {(a)};
			\end{tikzpicture}
			&
			\begin{tikzpicture}
			\begin{axis} [
			width=0.4\textwidth,%
			height = 5cm,%
			ybar,%
			bar width=10pt,
			title={ND 0 filter},%
			xtick = data,%
			symbolic x coords={pt1, pt2, pt3},%
			ylabel= {nsec},%
			ymin=0,ystep=0.2,ymax=2.5,%
			scaled y ticks = false,%
			ymajorgrids = true,
			legend style={at={(0.02,10)}},legend pos=north west]%
			\addplot table [x=pt, y=tau1] {./ND_data/nd0_2.csv}; \addlegendentry {tau 1},%
			\addplot table [x=pt, y=tau2]
			{./ND_data/nd0_2.csv}; \addlegendentry {tau 2}%
			\end{axis}
			\node at (-0.2, 4.0) {(b)};
			\end{tikzpicture}
			&\includegraphics[width=3cm]{nd0place}
			\begin{tikzpicture} [remember picture,overlay]
			\node at (-2.8, 3.0){(c)};
			\end{tikzpicture}
				
		\end{tabular}		
		\caption{Both exciton / biexciton recombination rate increased when the laser point headed to the center of the crystal. ND 0 filter is used. Tau 1 means the exciton recombination rate, and tau 2 mean the biexciton recombination rate.}	
		\label{fig:FIR107}
	\end{center}
\end{figure}

Figure 7은 exciton / biexciton recombination rate의 비율이 아닌, 값 자체를 히스토그램으로 나타낸 것으로, 두 값 모두가 결정의 중심쪽으로 갈수록 증가하고 있음을 관찰할 수 있다. (c)는 (a)와 (b)에서의 x축에 해당하는 점들의 위치를 표시하고 있다. (a)의 경우 TRPL 데이터가 점 6에서 점 4로 갈 때에 recombination rate가 증가하는 것을 볼 수 있다. pt 1, 2, 3의 경우에는 pt 3의 exciton recombination rate가 약간 감소하였으나, 그 이상으로 biexciton recombination rate가 증가했기 때문에 감소한 exciton recombination이 biexciton recombination으로 대체되었다고 할 수 있었다.
마찬가지로 (b)의 경우에도 TRPL 데이터가 점 1에서 2를 거쳐 3으로 갈 때에도 recombination rate가 증가하였다.
Figure 8은  Figure 7과 마찬가지로 결정의 중심부분으로 갈수록 recombination rate가 증가하는 모습을 보여준다. 하지만, 이때는 Figure 7과는 다르게 pt 9에서 pt 10으로 향할 때는 두 rate가 모두 감소하였는데, 이는 결정의 중심부에서 결정이 완성되지 못했음을 의미한다. 

\begin{figure}[H]
	\begin{center}
		\begin{tabular}{ccc}
			\begin{tikzpicture}
			\begin{axis} [
			width=0.4\textwidth,%
			height = 5cm,%
			ybar,%
			bar width=10pt,
			title={ND 0 filter},%
			xtick = data,%
			symbolic x coords={pt2, pt11, pt12},%
			ylabel= {nsec},%
			ymin=0,ystep=0.2,ymax=2.5,%
			scaled y ticks = false,%
			ymajorgrids = true,
			legend style={at={(0.02,10)}},legend pos=north west]%
			\addplot table [x=pt, y=tau1] {./ND_data/nd1_1.csv}; \addlegendentry {tau 1},%
			\pagestyle{empty}
			\addplot table [x=pt, y=tau2]
			{./ND_data/nd1_1.csv}; \addlegendentry {tau 2}%
			\end{axis}
			\node at (-0.2, 4.0) {(a)};
			\end{tikzpicture}
			&
			\begin{tikzpicture}
			\begin{axis} [
			width=0.4\textwidth,%
			height = 5cm,%
			ybar,%
			bar width=10pt,
			title={ND 0 filter},%
			xtick = data,%
			symbolic x coords={pt6, pt7, pt8, pt9, pt10},%
			ylabel= {nsec},%
			ymin=0,ystep=0.2,ymax=2.5,%
			scaled y ticks = false,%
			ymajorgrids = true,
			legend style={at={(0.02,15)}},legend pos=north west]%
			\addplot table [x=pt, y=tau1] {./ND_data/nd1_2.csv}; \addlegendentry {tau 1},%
			\addplot table [x=pt, y=tau2]
			{./ND_data/nd1_2.csv}; \addlegendentry {tau 2}%
			\end{axis}
			\node at (-0.2, 4.0) {(b)};
			\end{tikzpicture} &
			\includegraphics[width=3cm]{nd1place}
			\begin{tikzpicture} [remember picture,overlay]
			\node at (-2.8, 2.5){(c)};
			\end{tikzpicture}
		\end{tabular}
		\caption{Same tendency can also be seen when ND 1 filter is used.}	
		\label{fig:FIR109}
	\end{center}
\end{figure} 