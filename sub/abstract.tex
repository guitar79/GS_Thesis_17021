\maketitle  % command to print the title page with above variables
\setcounter{page}{1}
%---------------------------------------------------------------------
%                  영문 초록을 입력하시오
%---------------------------------------------------------------------
\begin{abstracts}     %this creates the heading for the abstract page
	\addcontentsline{toc}{section}{Abstract}  %%% TOC에 표시
	\noindent{
		Stars are born when matter from interstellar molecular clouds fall to the center to increase the mass of the protostar. Bipolar outflows are formed to remove the excess angular momentum of falling matter. Intensities of outflows are known to be in a close relationship with their bolometric luminosity and evolutionary stages. In this study, data from Institute for Radio Astronomy in the Millimeter Range (IRAM) 30$\,$m Telescope and Taeduk Radio Astronomy Observatory (TRAO) were used. IRAM data were used to map $^{12}\textrm{CO}$ J = 2 - 1 over Orion A molecular cloud. TRAO data were used to map $^{13}\textrm{CO}$ J = 1 - 0 over the same region. Outflows were observed and measured by drawing contour maps and line profiles of  red/blue shifted components. The correlation between a protostar's luminosity and outflow momentum flux have been confirmed. Also, outflows could be detected better if the energy level of the emission line is higher. 
	}
\end{abstracts}

\begin{abstractskor}
	페로브스카이트는 전하 수송 능력이 좋고 제조하기 쉽고 값싸다는 데에 장점을 두며 많은 광학 소자에서 응용되고 있다. 특히 이 물질은  태양전지에서 태양에너지를 이용해서 전하를 수송하는 역할로 쓰인다. 
	본 연구에서는 대표적인 페로브스카이트인 $\rm{CsPbBr_3}$  단결정을 비교적 최근에 발견된 방법인 PDMS-stamping 방법으로 만들고, 만들어진 결정이 $\rm{CsPbBr_3}$의 특성을 지니는지 확인한다. X-Ray diffraction (XRD) 을 통해 결정의 구조를 확인하고, Time-Resolved 
	Photoluminescence (TRPL) 분석을 통해 결정의 carrier dynamics를 확인할 수 있다. 이를 기존의 박막 형태의 $\rm{CsPbBr_3}$의 데이터와 이를 비교하였다.  이를 통해 PDMS-stamping 방식이 페로브스카이트를 만드는데 쓰일 수 있다는 것을 보일 수 있으며, 더 나아가 결정의 내부와 외부에서의 TRPL 데이터의 변화를 통해 결정의 순도에 대한 정성적인 예측을 할 수 있다.
	\\Key words: TRPL, $\rm{CsPbBr_3}$ single crystal, carrier lifetime
\end{abstractskor}
%----------------------------------------------
%   Table of Contents (자동 작성됨)
%----------------------------------------------
\cleardoublepage
\addcontentsline{toc}{section}{Contents}
\setcounter{secnumdepth}{3} % organisational level that receives a numbers
\setcounter{tocdepth}{3}    % print table of contents for level 3
\baselineskip=2.2em
\tableofcontents


%----------------------------------------------
%     List of Figures/Tables (자동 작성됨)
%----------------------------------------------
\cleardoublepage
\clearpage
\listoffigures	% 그림 목록과 캡션을 출력한다. 만약 논문에 그림이 없다면 이 줄의 맨 앞에 %기호를 넣어서 코멘트 처리한다.

\cleardoublepage
\clearpage
\listoftables  % 표 목록과 캡션을 출력한다. 만약 논문에 표가 없다면 이 줄의 맨 앞에 %기호를 넣어서 코멘트 처리한다.

%%%%%%%%%%%%%%%%%%%%%%%%%%%%%%%%%%%%%%%%%%%%%%%%%%%%%%%%%%%
%%%% Main Document %%%%%%%%%%%%%%%%%%%%%%%%%%%%%%%%%%%%%%%%
%%%%%%%%%%%%%%%%%%%%%%%%%%%%%%%%%%%%%%%%%%%%%%%%%%%%%%%%%%%
\cleardoublepage
\clearpage
\renewcommand{\thepage}{\arabic{page}}
\setcounter{page}{1}



