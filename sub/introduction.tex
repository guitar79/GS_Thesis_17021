%-----------------------------------------------------
%  Introduction
%-----------------------------------------------------

\section{Introduction}

Stars are formed in molecular clouds by gravitational accretion. In the early stages of star formation, young stellar objects(YSOs) are still embedded in the molecular clouds, increasing its mass and temperature by accretion of interstellar medium around it. Since the angular momentum is conserved while matter is accreted, matter near the surface of the protostar spins quickly, which prevents more accretion. Since angular momentum is removed by jets called bipolar outflows, outflows are observed with size proportional to the mass accreted to the protostar\cite{bontemps1996evolution}. 
It is already known that the accretion rate and luminosity correlate to each other \cite{kang2013outflow}. The outflow force decreases as protostars evolve from Class 0 to Class I, which means the strength with which the protostar pulls interstellar matter decreases as time passes. 

In this study, I identified the protostars and their outflows of the Orion A Cloud. Aso et al. \cite{aso2000dense} made observations of the $^{12}\textrm{CO (J = 1 - 0)}$ emission and identified 9 CO outflows. Also, Takahashi et al. \cite{takahashi2008millimeter} made observations of the $^{12}\textrm{CO (J = 3 - 2)}$ emmision lines and identified 15 outflows. First, I selected the protostars for which the outflows can be detected from the Spitzer and the Herchel catalogues \cite{megeath2012spitzer, furlan2016herschel}. By using different data sets observed by different observatories and different wavelengths, I identified the outflows. I rechecked the correlation between the outflow force and its bolometric luminosity. Also, I compared the outflow momentum flux calculated using different emission lines. \\
